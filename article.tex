
%Document generated using DScaffolding from https://www.mindmeister.com/1192295427
\documentclass{article}
\usepackage[utf8]{inputenc}

\newcommand{\todo}[1] {\iffalse #1 \fi} %Use \todo{} command for bringing ideas back to the mind map

\title{StrategicWriting (2)}
\author{}

\begin{document}

\maketitle
      

\section{Introduction}



%Describe the practice in which the problem addressed appears
A lot of authors....There is a growing body of literature that recognises the importance of Scientific Writing in Design Science Research. A key aspect of Scientific Writing in Design Science Research is that it is Different with respect to ordinary writing. With respect to this, it has been reported that Requires rigour \cite{JohannessonAnIntroduction}, Articles follow specific structures \cite{Turbek2016} \cite{JohannessonAnIntroduction} \cite{psychologicalscience.org} \cite{Kanoksilapatham2012} and Space limitations \cite{Gregor2013} \cite{Hsu2004}. Apart from that, Pressure to publish is a fundamental property of Scientific Writing in Design Science Research \cite{CoylarBecomingwriters} \cite{Turbek2016} \cite{Sorensen1994} \cite{Kanoksilapatham2012} \cite{Mizumoto2017} \cite{Cotterall2011}. Scientific Writing in Design Science Research encompasses different activities: Writing the introduction, Receiving feedback and Collecting evidences. As for Writing the introduction, it has been described as Difficult \cite{Hsu2004} \cite{Hsu2004} \cite{Hsu2004} \cite{PeatScientificWriting.} \cite{Cotterall2011} and Important \cite{Gregor2013} \cite{Turbek2016} \cite{Turbek2016} \cite{SwalesTheWriting} \cite{Khaw2017} \cite{Kanoksilapatham2012}. It is conducted using Word processors (such as LaTeX, Microsoft Word and Overleaf) and Graphic editors. As regards Collecting evidences, it is conducted using Reference managers (such as Zotero and Mendeley). 
    
%Describe the practical problem addressed, its significance and its causes
Research has shown that a major problem within Scientific Writing in Design Science Research is that student belated writing \cite{Wellington2010} \cite{Badley2009} \cite{Turbek2016}. This problem is of particular concern as it is now well established that it can lead to poor reuse / impact and dropout \cite{Wellington2010} \cite{Itua2014}. Causes can be diverse: (1) "Writing" regarded as an ancillary activity \cite{Turbek2016} \cite{Turbek2016}, (2) Lack of opportunities for developing writing skills \cite{Cotterall2011}, (3) Not exposed to the conventions of research discourse \cite{Elton2010} \cite{Badenhorst2015} \cite{Wellington2010} \cite{Cotterall2011} \cite{Itua2014}, (4) Supporting Evidences? \cite{Hsu2004} \cite{Badenhorst2015} \cite{Badenhorst2015} \cite{Cotterall2011} \cite{Itua2014} \cite{Cameron2009} \cite{Lindsay1854}, (5) Blank page syndrome \cite{Lindsay1854} \cite{Sorensen1994} \cite{Wellington2010} \cite{Cameron2009}, (6) lack of confidence \cite{Carter2012}, (7) Lack of a publication schema for DSR \cite{Gregor2013} \cite{Gregor2013} and (8) Limited English abilities \cite{Mizumoto2017} \cite{Cotterall2011} \cite{Anthony2003}. 
    
%Summarise existing research including knowledge gaps and give an account for similar and/or alternative solutions to the problem
Existing research has tackled these causes. Badenhorst et al. addressed the Not exposed to the conventions of research discourse \cite{Badenhorst2015}. Elton et al. addressed the Not exposed to the conventions of research discourse \cite{Elton2010}. Gregor et al. addressed the Lack of a publication schema for DSR \cite{Gregor2013}. Johannesson et al. addressed the Lack of a publication schema for DSR \cite{JohannessonAnIntroduction}. Mizumoto et al. addressed the Limited English abilities \cite{Mizumoto2017}. 
    
%Formulate goals and present Kernel theories used as a basis for the artefact design
In this work, we address 4 main causes: "Writing" regarded as an ancillary activity, Not exposed to the conventions of research discourse, Blank page syndrome and Limited English abilities. To lessen these causes, we resort to Knowledge-transforming model for composition and Scaffolding. 
    
%Describe the kind of artefact that is developed or evaluated
This article presents an artefact named DScaffolding. This artefact is a round-trip editor based on mind-mapping (MindMeister) & text processors (LaTeX). 
    
%Formulate research questions
In summary, along Wieringa's template \cite{Wieringa2014}, this paper's design problem can be enunciated as follows: 
improve student belated writing
by designing a(n) round-trip editor based on mind-mapping (MindMeister) & text processors (LaTeX)
that satisfies Make the thinking visible, Structure the information of a DSR project, Manage the information flow between the writing and the thinking processes, Make the expected structure of the research article introduction explicit, Generate an article scaffold from the project documentation and Make the user aware of appropriate expressions
in order to help PhD students achieve Get work published and PhD supervisors achieve Increase Impact. 
    
%Summarize the contributions and their significance

      
%Overview of the research strategies and methods used
This article has followed a Design Science Research approach.

%Describe the structure of the paper
The remainder of the paper is structured as follows: 

%Optional - illustrate the relevance and significance of the problem with an example
    
      
\bibliographystyle{unsrt}
\bibliography{references}

\end{document}
    